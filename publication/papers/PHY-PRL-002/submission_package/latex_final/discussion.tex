These four experimental predictions provide distinct signatures of the Universe Ontology theoretical framework that can be tested using current or near-term quantum experimental capabilities. They target different aspects of quantum physics: causal structures, non-local correlations, phase transitions, and coherence dynamics.

The predictions are specifically designed to differentiate UO theory from conventional quantum mechanics and other competing theories. In particular:

\begin{enumerate}
\item The specific form of invariance under XOR-SHIFT transformations is unique to the UO framework
\item The non-local correlation preservation under XOR operations differs from standard Bell inequality predictions
\item The critical exponent at the predicted phase transition would serve as a distinctive signature
\item The coherence oscillation pattern under sequential XOR operations provides a clear fingerprint of the theory
\end{enumerate}

These experiments collectively form a comprehensive test suite for the foundation of the Universe Ontology theory. They are designed to be independently verifiable across different quantum experimental platforms, ensuring robust validation or falsification of the theory's core predictions. 