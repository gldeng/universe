The Universe Ontology framework employs information theoretic methods to derive physical principles from fundamental information operations. Here we outline our methodological approach and the mathematical tools employed.

\subsection{Information Operations Formalism}

We develop a formal system based on two primitive operations:

\begin{itemize}
    \item \textbf{XOR Operation ($\xor$)}: Represents information difference detection between states
    \item \textbf{SHIFT Operation ($\shift$)}: Represents information displacement or perspective transformation
\end{itemize}

These operations act on information states in an abstract state space. The operations satisfy several algebraic properties:

\begin{align}
A \xor A &= 0 \\
A \xor 0 &= A \\
A \xor B &= B \xor A \\
(A \xor B) \xor C &= A \xor (B \xor C) \\
\shift(A \xor B) &= \shift(A) \xor \shift(B)
\end{align}

\subsection{Quantum-Classical Boundary Model}

The quantum domain ($\quantumdomain$) and classical domain ($\classicdomain$) are related by:

\begin{equation}
\classicdomain = \quantumdomain \xor \shift(\quantumdomain)
\end{equation}

This relationship is used to derive quantum measurement effects, wave-particle duality, and the emergence of classical reality.

\subsection{Verification Methods}

We employ three complementary approaches for verification:

\begin{enumerate}
    \item \textbf{Mathematical consistency}: Proving that our framework is internally consistent and mathematically sound
    \item \textbf{Explanatory power}: Demonstrating that known physical phenomena can be derived from our principles
    \item \textbf{Predictive testing}: Formulating testable predictions that differentiate our theory from existing frameworks
\end{enumerate}

For experimental verification, we focus on quantum interference phenomena where the theory predicts subtle deviations from standard quantum mechanics at specific scales. Our simulations use numerical methods to solve the state evolution equations in systems where quantum and classical domains interact.

\subsection{Computational Methods}

We implemented computational simulations using custom Python code with NumPy and SciPy libraries. The core algorithm tracks the evolution of quantum states undergoing successive XOR-SHIFT operations. The simulation code and data are available in the supplementary materials.
