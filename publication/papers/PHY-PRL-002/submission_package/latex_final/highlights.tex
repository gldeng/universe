\documentclass[11pt,letterpaper]{article}
\usepackage[utf8]{inputenc}
\usepackage{graphicx}
\usepackage{hyperref}
\usepackage{geometry}
\usepackage{enumitem}
\usepackage{titlesec}
\geometry{letterpaper, margin=1in}

\titleformat{\section}
  {\normalfont\Large\bfseries}
  {}
  {0em}
  {}

\begin{document}

\begin{center}
\Large\textbf{Research Highlights: Micro-physics Verification Predictions of Universe Ontology}
\end{center}

\section*{Key Contributions}

\begin{enumerate}[leftmargin=*]
\item \textbf{Falsifiable Predictions}: Presents five specific, experimentally testable predictions derived from Universe Ontology that diverge from standard quantum theory.

\item \textbf{Novel Quantum Boundary Theory}: Develops a mathematically rigorous framework connecting quantum and classical domains through information operations.

\item \textbf{Experimental Pathways}: Provides concrete experimental designs to test each prediction using current or near-future technology.

\item \textbf{Mathematical Innovation}: Derives new physical formulas from minimal axioms based on XOR and SHIFT information operations.

\item \textbf{Multi-domain Impact}: Connects fundamental theory to practical applications across quantum technology, gravitational physics, and quantum information science.
\end{enumerate}

\section*{Significance to the Field}

\begin{itemize}[leftmargin=*]
\item First comprehensive set of experimentally verifiable predictions from an information-theoretic approach to fundamental physics

\item Proposes resolution pathways for long-standing issues in quantum measurement, non-locality, and quantum-gravity integration

\item Establishes quantitative boundaries for quantum effects based on information content rather than traditional physical parameters

\item Opens new experimental directions for testing quantum foundations with precise, quantitative predictions
\end{itemize}

\section*{Technological Applications}

\begin{itemize}[leftmargin=*]
\item Enhanced precision in quantum technology through better understanding of decoherence mechanisms

\item New protocols for quantum information transfer with optimized efficiency based on information-theoretic limits

\item Improved designs for quantum sensors operating near gravitational gradient thresholds

\item Potential applications in quantum communication, computation, and sensing technologies
\end{itemize}

\end{document} 