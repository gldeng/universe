Physics has traditionally progressed through the discovery of fundamental laws and particles, but has reached impasses at several critical boundaries. The lack of unification between quantum mechanics and general relativity, the measurement problem in quantum mechanics, and the unexplained nature of dark matter and dark energy all point to potential limitations in our current conceptual frameworks \cite{Wheeler1990}.

Information has increasingly been recognized as a fundamental concept in physics \cite{vonBaeyer2003, Vedral2010}. Wheeler's ``It from Bit'' proposal suggested that information might be more fundamental than physical entities, and recent work in quantum foundations, black hole thermodynamics, and holographic principles has strengthened this perspective \cite{Rovelli2015}.

The Universe Ontology framework presented here proposes a radical shift: physical reality emerges from fundamental information operations rather than from particles, fields, or spacetime. We propose that two primitive operations—XOR ($\xor$) and SHIFT ($\shift$)—form the basis of all physical phenomena. The XOR operation represents information difference detection, while the SHIFT operation represents information displacement or perspective change.

In this framework, the quantum domain ($\quantumdomain$) and classical domain ($\classicdomain$) are related by:

\begin{equation}
\classicdomain = \quantumdomain \xor \shift(\quantumdomain)
\end{equation}

This relationship explains the quantum-classical boundary and provides a unified mathematical framework for describing physical phenomena across scales. The theory naturally accounts for quantum measurement, wave-particle duality, non-locality, and derives several known physical laws from information principles.

The Universe Ontology framework offers several advantages over existing theories: it is conceptually simpler, more unified, and makes novel predictions that can be tested experimentally. In the following sections, we introduce the mathematical formalism, derive key results, compare with existing theories, and propose experimental tests. 