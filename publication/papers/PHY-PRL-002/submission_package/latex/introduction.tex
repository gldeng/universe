The unification of quantum and classical physics remains one of the most significant challenges in theoretical physics. The Universe Ontology (UO) theory~\cite{UOTS} proposes a novel approach to this problem by postulating that fundamental information operations—specifically XOR and SHIFT—form the underlying basis for all physical phenomena. While the theoretical framework has demonstrated mathematical consistency and explanatory power, empirical verification is essential for scientific validation.

This paper focuses on experimentally testable predictions derived from the Quantum XOR Causal Invariance theory~\cite{QXCIT}, a key component of the UO framework that describes causal relationships in quantum systems through XOR operations. We present four specific experimental predictions, each targeting different aspects of quantum physics, and describe experimental protocols that could verify or falsify these predictions. 