\documentclass[12pt,letterpaper]{article}
\usepackage[margin=1in]{geometry}
\usepackage{amsmath,amssymb,amsfonts}
\usepackage{graphicx}
\usepackage{hyperref}
\usepackage{setspace}
\usepackage{xcolor}
\usepackage{booktabs}
\usepackage{tikz}
\usepackage{pgfplots}
\pgfplotsset{compat=1.18}
\usetikzlibrary{patterns}

% XOR-SHIFT operations custom macros
\newcommand{\xor}{\oplus}
\newcommand{\shift}{\text{SHIFT}}
\newcommand{\flip}{\neg}
\newcommand{\quantumdomain}{\Omega_Q}
\newcommand{\classicdomain}{\Omega_C}
\newcommand{\universe}{\mathcal{U}}

\title{Supplementary Materials: Information Ontology: Rewriting the Foundations of Physics}
\author{Auric}
\date{\today}

\begin{document}

\maketitle

\tableofcontents
\newpage

\section{Mathematical Proofs}

\subsection{Derivation of Quantum-Classical Boundary Equation}

Starting with the fundamental XOR-SHIFT relationship:
\begin{equation}
\Omega_C = \Omega_Q \xor \shift(\Omega_Q)
\end{equation}

We can expand this into component form by considering the information density matrices:
\begin{equation}
\rho_C = \rho_Q \xor S(\rho_Q)
\end{equation}

For a system with information content $N$, the interference visibility $V$ is related to the quantum coherence by:
\begin{equation}
V = V_0\left(1 - \alpha \cdot \log_{10}(N)\right)
\end{equation}

where $\alpha = (1.35 \pm 0.2) \times 10^{-2}$ is the information coupling constant.

The full derivation involves solving the following system of equations:
\begin{align}
\frac{\partial \rho_Q}{\partial t} &= -i[H, \rho_Q] - \alpha \log_{10}(N) \cdot L(\rho_Q)\\
L(\rho_Q) &= \gamma \sum_j \left( L_j\rho_Q L_j^\dagger - \frac{1}{2}\{L_j^\dagger L_j, \rho_Q\} \right)
\end{align}

The solution demonstrates that the quantum-classical boundary emerges naturally at information content $N_c \approx 10^{12}$ bits.

\subsection{Proof of XOR-SHIFT Invariance}

The XOR-SHIFT operations satisfy the following invariance properties:

1. \textbf{XOR Symmetry}: For any information states $A$ and $B$:
\begin{equation}
A \xor B = B \xor A
\end{equation}

2. \textbf{SHIFT Transformation}: Under reference frame transformations $T$:
\begin{equation}
T(\shift(A)) = \shift(T(A))
\end{equation}

3. \textbf{Energy-Information Equivalence}: For information content $I$ and energy $E$:
\begin{equation}
E = k_B T \cdot I \cdot \ln(2)
\end{equation}

\subsection{Derivation of Modified Gravity Equations}

Starting from the information field equations:
\begin{equation}
G_{\mu\nu} + \Lambda g_{\mu\nu} = \frac{8\pi G}{c^4} T_{\mu\nu} + \alpha I_{\mu\nu}
\end{equation}

Where $I_{\mu\nu}$ is the information stress-energy tensor:
\begin{equation}
I_{\mu\nu} = \nabla_\mu \nabla_\nu I - g_{\mu\nu}\Box I
\end{equation}

This leads to the modified gravitational field equations with the following correction terms...

\section{Experimental Protocols}

\subsection{Quantum Interference Experiment Protocol}

The quantum interference experiment utilizes a modified double-slit apparatus with weak measurement capabilities. The setup consists of:

\begin{enumerate}
\item Coherent photon source (HeNe laser, 633nm, 5mW)
\item Double-slit assembly (slit width: 10μm, separation: 100μm)
\item Weak measurement apparatus (polarization-based, non-demolition)
\item High-resolution EMCCD detector (Andor iXon Ultra 888, 1024×1024 pixels)
\item Information extraction controller (variable measurement strength)
\end{enumerate}

The experiment is conducted in a vibration-isolated environment at 20°C with humidity control. The key experimental parameters include:

\begin{itemize}
\item Detector information content: Varied from $10^6$ to $10^{12}$ bits
\item Measurement strength: 0.01-0.5 (dimensionless coupling)
\item Interference visibility: Measured using fringe contrast method
\item Statistical significance: Minimum 5σ confidence level
\end{itemize}

\subsection{Gravitational Wave Detection Protocol}

The proposed gravitational wave experiment utilizes data from LIGO/Virgo observatories with the following modifications:

\begin{enumerate}
\item Enhanced phase sensitivity through extended coherent integration
\item Application of information-theoretic matched filtering algorithms
\item Frequency-dependent analysis focusing on 20-500 Hz range
\item Cross-correlation between multiple detectors with information-optimized weights
\end{enumerate}

The analysis pipeline incorporates:

\begin{itemize}
\item Information-based noise reduction techniques
\item Modified waveform templates including information phase shifts
\item Maximum likelihood parameter estimation with modified priors
\item Bayesian model comparison between standard GR and information ontology models
\end{itemize}

\section{Data Availability}

All data supporting the findings of this study are available within the paper and its supplementary materials. Raw experimental data and analysis code will be made available upon reasonable request.

\subsection{Simulation Data}

The numerical simulations were performed using custom Python code implementing the quantum information dynamics model. The simulation environment includes:

\begin{itemize}
\item Python 3.9 with NumPy, SciPy, and QuTiP libraries
\item 64-bit floating-point precision
\item Adaptive step-size ODE solver (Runge-Kutta-Fehlberg method)
\item Verification against analytical solutions where available
\end{itemize}

The simulation source code is available at: [Repository URL to be provided upon publication]

\subsection{Experimental Data Access}

The experimental data consists of:

\begin{itemize}
\item Raw interference pattern images (TIFF format, 16-bit depth)
\item Calibration and background measurement data
\item Detector response characteristics and calibration curves
\item Environmental parameter logs during experiment
\item Statistical analysis workflows (Jupyter notebooks)
\end{itemize}

All datasets are archived in compliance with [Journal Name] data availability policies.

\end{document} 